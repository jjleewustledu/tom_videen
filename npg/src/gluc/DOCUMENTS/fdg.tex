\documentclass[12pt]{article}    % Specifies the document style.
\setlength{\oddsidemargin}{+0.05\oddsidemargin}
\setlength{\textwidth}{6.4in}
\setlength{\textheight}{9.0in}
\setlength{\topmargin} {-0.1in}
\title{ dummy}
\begin{document}
\newpage
\begin{center}
\Large
{ FDG MODEL }
\end{center}
\sf
\begin{picture}(500,130)(0,0)
\thicklines
\large
\multiput(50,60)(150,0){3}{\framebox(50,50){  }}
\put(100,90){\vector(+1,0){100}}
\put(250,96){\vector(+1,0){100}}
\put(350,84){\vector(-1,0){100}}
\put(225,60){\vector(0,-1){50}}
\put(145,98){\(K_{1}\)}
\put(230,35){\(k_{2}\)}
\put(292,103){\(k_{3}\)}
\put(292,70){\(k_{4}\)}
\put(63,85){\(C_{a}(t)\)}
\put(223,85){\(q_{2}\)}
\put(373,85){\(q_{3}\)}
\end{picture}
\begin {description}
\item [Compartment 1 ] -  FDG in vascular space 
\item [Compartment 2 ] -  FDG in extravascular space 
\item [Compartment 3 ] -  FDG-6-P in extravascular space
\end {description}
\vspace{0.1in}
The following set of differential equations describes the kinetics of
activity in the two extravascular compartments.
\vspace {0.15in}
\begin {eqnarray}
 \frac {dq_{2}(t)}{dt} & = & K_{1}C_{a}(t) -(k_{2} +  k_{3})q_{2}(t) +
       k_{4}q_{3}(t) \\
 \frac {dq_{3}(t)}{dt} & = & k_{3}q_{2}(t) - k_{4}q_{3}(t) 
\end{eqnarray}

  Solutions to the differential equations are  as follows:
\vspace {0.15in}
\begin {eqnarray}
 q_{2}(t) & = & \frac{K_{1}}{\beta -\alpha}[\,( k_{4} - \alpha) e ^{-\alpha t}
 + (\beta - k_{4}) e^{-\beta t}\,] * C_{a}(t) \\
 q_{3}(t)  & = & \frac{ k_{3}K_{1}} {\beta - \alpha}[\,e^{-\alpha t} 
  - e^{-\beta t}\, ] * C_{a}(t)  \\
  \beta   & = & \Bigl [\,k_{2} + k_{3} + k_{4} + \sqrt {(k_{2} +  k_{3} 
   + k_{4})^{2} - 4 k_{2}k_{4}}\; \Bigr ] /2  \\
  \alpha  & = & \Bigl [ \,k_{2} + k_{3} + k_{4} - \sqrt {(k_{2} +  k_{3} 
   + k_{4})^{2} - 4 k_{2}k_{4}} \; \Bigr ] /2  
\end {eqnarray}
 where the * denotes convolution.
   Total PET activity (PET counts/ml ) in a region of interest is
given by 
\begin {eqnarray}
  q_{pet}(t)  & = & V_{1}C_{a}(t)  + q_{2}(t) + q_{3}(t)  
\end {eqnarray}
where  $V_{1}$ is the fractional blood volume.  
Values for  the 4 parameters $ K_{1}, k_{2}, k_{3} $,and $  k_{4} $  are
estimated by a nonlinear least-squares algorithm based on Marquaradt's method
\newpage
\end{document}             % End of document.

