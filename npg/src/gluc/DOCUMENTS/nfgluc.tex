\documentclass[11pt]{article}    % Specifies the document style.
\setlength{\oddsidemargin}{+0.05\oddsidemargin}
\setlength{\textwidth}{6.4in}
\setlength{\textheight}{8.8in}
\begin{document}
\newpage
\noindent Joanne Markham   \hspace{4.0in} \today\
\Large
\sf
\begin{center}
NO FLOW GLUCOSE MODEL FOR NEUROLOGY 
\end{center}
\large
\sf
\begin{picture}(500,140)(0,0)
\thicklines
\large
\multiput(20,60)(100,0){4}{\framebox(50,50){  }}
\put(70,85){\vector(+1,0){50}}
\put(170,85){\vector(+1,0){50}}
\put(270,85){\vector(+1,0){50}}
\put(145,60){\vector(0,-1){50}}
\put(350,60){\vector(0,-1){50}}
\put( 88,96){\(k_{21}\)}
\put(188,96){\(k_{32}\)}
\put(288,96){\(k_{43}\)}
\put(354,35){\(k_{04}\)}
\put(154,35){\(k_{12}\)}
\put(45, 90){1}
\put(145,90){2}
\put(245,90){3}
\put(345,90){4}
\put(35,75){\(C_{a}(t)\)}
\put(140,75){\(q_{2}\)}
\put(240,75){\(q_{3}\)}
\put(340,75){\(q_{4}\)}
\end{picture}
\begin {eqnarray}
  q_{1}(t) & =  & V_{B} \; C_{a}(t)   \\
 \frac {dq_{2}(t)}{dt} & = & k_{21}\;q_{1}(t) -(k_{12} +  k_{32})\;q_{2}(t) \\
 \frac {dq_{3}(t)}{dt} & = & k_{32}\;q_{2}(t) - k_{43}\;q_{3}(t) \\
 \frac {dq_{4}(t)}{dt} & = & k_{43}\;q_{3}(t) - k_{04}\; q_{4}(t)  \\
  k_{22}  & = & k_{12} +  k_{32} 
\end{eqnarray}

  Solutions to the differential equations are  as follows:
\vspace {0.15in}
\begin {eqnarray}
  q_{1}(t) & =  & V_{B} \; C_{a}(t)   \\
  q_{2}(t) & =  & V_{B}\;  k_{21} \;e^{- k_{22} t}  \otimes C_{a}(t)  \\
 q_{3}(t) & = &  \frac { V_{B}\;   k_{21}\; k_{32}}{(k_{22} - k_{43})} \;
 \bigl [ \;  e ^{-k_{43} t} - e ^{-k_{22} t} \bigr ] \otimes C_{a}(t)  \\
 q_{4}(t) & = &  \ V_{B}\; k_{21}\; k_{32}\; k_{43} \Biggl \{  \biggl [
\frac  { e ^{-k_{22} t} } { (k_{04} - k_{22}) (k_{43} - k_{22}) }  \; \biggr ] +
\biggl [ \frac {e ^{-k_{43} t} } {(k_{22} - k_{43}) (k_{04} - k_{43})} \; \biggr ] + \nonumber \\ 
& & \biggl [ \frac {e ^{-k_{04} t} } {(k_{22} - k_{04}) (k_{43} - k_{04})} \; \biggr ]
    \Biggr \}  \otimes C_{a}(t)   
\end {eqnarray}
 where the $\otimes$  denotes convolution.
   Total PET activity (PET counts/ml ) in a region of interest is
given by 
\begin {eqnarray}
  q_{pet}(t)  & = & V_{B}C_{a}(t)  + q_{2}(t) + q_{3}(t)  +  q_{4}(t)
\end {eqnarray}
where  $V_{B}$ is the fractional blood volume.  
\newpage
   The equation for the  extravascular activity can be written as
\begin {eqnarray}
q_{e}(t)  & =  &  q_{2}(t) + q_{3}(t)  +  q_{4}(t) \\
q_{e}(t)  & =  &   [ A  e ^{-k_{22} t} + B e ^{-k_{43} t} + C e ^{-k_{04} t}] \otimes C_{a}(t) \\
 A  & =  &  V_{B}\;  k_{21} \biggl [ 1 +  \frac {k_{32}  } {(k_{43} - k_{22})}
 + \frac {k_{32}  k_{43} }  {(k_{43} - k_{22})  (k_{04} - k_{22}) }  \\
  B  & =  &   \frac { V_{B} \; k_{21}\; k_{32} \; k_{04} } {(k_{22} - k_{43}) (k_{04} - k_{43}) } \\
  C   & =  &  \frac { V_{B} \; k_{21} \; k_{32} \;k_{43}} {(k_{22} - k_{04}) (k_{43} - k_{04})} 
\end {eqnarray}

\vspace {0.35in}
\begin{center}

\begin{tabular}{|l|l|l|}
\hline
  VARIABLE  &  DESCRIPTION   & UNITS \\
\hline
\     &   &  \\
\(C_{a}(t)\) & concentration of activity in blood & (PET counts/ml blood)\\
\      &  &   \\
\(q_{i}(t)\)  &  activity in compartment i normalized 
  & (PET counts/ml) \\
    & to PET volume &  \\
\  &  &   \\
\(K_{1}\) & rate of movement of FDG from vascular & (ml blood/ml sec)\\
    &  to extravascular space  = $V_{B}k_{21}$ &   \\
\     &    &   \\
\(k_{ij}\) &  rate constant for kinetics to compartment i from compartment j   & (1/sec) \\
    &  as shown in diagram &  \\
\   &   &  \\
\(V_{B} \) & blood volume & (ml blood/ml ) \\
\  &   &    \\
\hline
\end {tabular}
\end {center}

\begin{thebibliography}{2}
\bibitem{[1]}
  Huang, S-C., M. E. Phelps, E. J. Hoffman, K. Sideris, C. J. Selin, and
   D. E. Kuhl.
  Noninvasive determination of local cerebral metabolic
   rate of glucose in man. 
  Am. J. Physiol. 238:E69-E82, 1980.
\bibitem{[2]}

   Marquardt, D. W., An algorithm for least-square estimation of 
     nonlinear parameters. J.  SIAM 11:431-441, 1963.
\end{thebibliography}

\newpage
\end{document}             % End of document.

